\section{Summary}\label{sect:summary}
We have described today's situation with regards to objective video quality assessment, where available tools are either too expensive for most users, or provide results that correlates poorly with subjective test results. We have given an overview of the history of standardization of no-, reduced- and full-reference objective video quality assessment tools as well as a description of the difference between these approaches. Based on the test results of the standardized models we have argued for why we have chosen the PEVQ model which we will implement, and we have given a brief description of design decisions and plans for our own implementation. Finally we have given an overview of our current progress on the implementation, including a description of the results it provides when testing a video sequence with certain degradation.

\subsection{Future work}\label{sect:future work}
The main goal and what we will begin working on is implementing the PEVQ model exactly as it is described in the standardization ITU-T Rec. J.247. This work has begun, and will continue over the last half of 2015. As mentioned in section~\ref{sect:cts} the standardized PEVQ model we are implementing does not support HD resolution. Another research topic we must keep in mind once our implementation is done is therefore extending our solution to support resolutions above VGA. We also mentioned in~\ref{sect:coding} that speedup using a Graphics Processing Unit(GPU) can potentially be significant, and we would like to look at the possibility of performing parts of the calculation in the PEVQ model on a GPU.
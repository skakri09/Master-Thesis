\section{Objective video quality assessment}\label{sect:ovqa}
As we have been able to see from the history of objective video quality assessment, the approach with highest consistency compared to the subjective counterpart is the full reference (FR) version. This is however also the approach that requires the most background data for any video sequence in order to give results. The reduced reference (RR) and no reference (NR) methods can be executed without access to the full source material. In order to better understand our decision to choose a FR model the next three sections will give an overview of the three approaches to objective video quality assessment. To read more about the differences to the methods mentioned in the following three sections you can see S. Chikkerur and V. Sundarams paper in\cite{5710601}.

%missing a reliable/good source for the sections
\subsection{No reference objective video quality assessment}\label{sect:nr}
NR points to the fact that there is no reference to the original source material when measuring and determining the quality of the video. This means that the assessment is done purely by analysing the degraded signal. In an environment where there is no access to any source material the only possible objective approach will be a NR method. However as we could see in section~\ref{sect:history} the performance of NR methods are to this date significantly lower compared to that of RR and FR approaches.


\subsection{Reduced Reference objective video quality assessment}\label{sect:rr}
RR is a middle ground between NR and FR in that it has access to certain information about the source signal. The information can be transmitted where there is limited transmission capacity and therefore limited access to the reference signal. This limited extra information about the source signal can still be enough to provide better results than a NR approach, and as we saw in section~\ref{sect:history} several standards have been approved for RR methods. 

%as could be seen in history this has some better accuracy, and standard(s?) have been approved.

\subsection{Full Reference objective video quality assessment}\label{sect:fr}
FR is the final approach where the entire source signal is available together with the degraded signal. This means we can compare the two signals as detailed as on a per pixel basis and draw a conclusion from the difference between the two signals. Considering all data is available during the measurement it makes sense that a FR technique will be most accurate, and as we could see from section~\ref{sect:history} this has also been the case in tests designed by VQEG. 
